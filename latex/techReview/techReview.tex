\documentclass[journal,compsoc, 10pt, draftclsnofoot, onecolumn]{IEEEtran}

\usepackage{graphicx}
\usepackage{amssymb}
\usepackage{amsmath}
\usepackage{amsthm}
\usepackage{tabularx}
\usepackage{graphicx}

\newcommand{\subparagraph}{}
\usepackage{titlesec}

\usepackage{alltt}
\usepackage{float}
\usepackage{color}
\usepackage{url}

\usepackage{balance}
\usepackage[TABBOTCAP, tight]{subfigure}
\usepackage{enumitem}
\usepackage{pstricks, pst-node}

\usepackage{cite}
\usepackage{listings}

\usepackage[margin=0.75in]{geometry}
\geometry{textheight=8.5in, textwidth=6in}

\renewcommand{\familydefault}{\sfdefault}


\newlength\tindent
\setlength{\tindent}{\parindent}
\setlength{\parindent}{0pt}
\renewcommand{\indent}{\hspace*{\tindent}}

\newcommand{\cred}[1]{{\color{red}#1}}
\newcommand{\cblue}[1]{{\color{blue}#1}}

\newcommand{\namesigdate}[2][6cm]{%
  \begin{tabular}{@{}p{#1}@{}}
    #2 \\[0.5\normalbaselineskip] \hrule \\[0pt]
    {\small \textit{Signature}} \\[0.5\normalbaselineskip] \hrule \\[0pt]
    {\small \textit{Date}}
  \end{tabular}
}


\usepackage{hyperref}
\usepackage{geometry}

\lstset{
language=C,
basicstyle=\ttfamily,
commentstyle=\color{blue},
keywordstyle=\color{green},
numberstyle=\color{red},
stringstyle=\color{orange}
}

\def\nameD{Devin Foulger}
\def\nameH{Hector Trujillo}
\def\nameB{Bryan Liauw}

\usepackage{fancyvrb}
\usepackage{color}
\usepackage[latin1]{inputenc}


\makeatletter
\def\PY@reset{\let\PY@it=\relax \let\PY@bf=\relax%
    \let\PY@ul=\relax \let\PY@tc=\relax%
    \let\PY@bc=\relax \let\PY@ff=\relax}
\def\PY@tok#1{\csname PY@tok@#1\endcsname}
\def\PY@toks#1+{\ifx\relax#1\empty\else%
    \PY@tok{#1}\expandafter\PY@toks\fi}
\def\PY@do#1{\PY@bc{\PY@tc{\PY@ul{%
    \PY@it{\PY@bf{\PY@ff{#1}}}}}}}
\def\PY#1#2{\PY@reset\PY@toks#1+\relax+\PY@do{#2}}

\expandafter\def\csname PY@tok@gd\endcsname{\def\PY@tc##1{\textcolor[rgb]{0.63,0.00,0.00}{##1}}}
\expandafter\def\csname PY@tok@gu\endcsname{\let\PY@bf=\textbf\def\PY@tc##1{\textcolor[rgb]{0.50,0.00,0.50}{##1}}}
\expandafter\def\csname PY@tok@gt\endcsname{\def\PY@tc##1{\textcolor[rgb]{0.00,0.25,0.82}{##1}}}
\expandafter\def\csname PY@tok@gs\endcsname{\let\PY@bf=\textbf}
\expandafter\def\csname PY@tok@gr\endcsname{\def\PY@tc##1{\textcolor[rgb]{1.00,0.00,0.00}{##1}}}
\expandafter\def\csname PY@tok@cm\endcsname{\let\PY@it=\textit\def\PY@tc##1{\textcolor[rgb]{0.25,0.50,0.50}{##1}}}
\expandafter\def\csname PY@tok@vg\endcsname{\def\PY@tc##1{\textcolor[rgb]{0.10,0.09,0.49}{##1}}}
\expandafter\def\csname PY@tok@m\endcsname{\def\PY@tc##1{\textcolor[rgb]{0.40,0.40,0.40}{##1}}}
\expandafter\def\csname PY@tok@mh\endcsname{\def\PY@tc##1{\textcolor[rgb]{0.40,0.40,0.40}{##1}}}
\expandafter\def\csname PY@tok@go\endcsname{\def\PY@tc##1{\textcolor[rgb]{0.50,0.50,0.50}{##1}}}
\expandafter\def\csname PY@tok@ge\endcsname{\let\PY@it=\textit}
\expandafter\def\csname PY@tok@vc\endcsname{\def\PY@tc##1{\textcolor[rgb]{0.10,0.09,0.49}{##1}}}
\expandafter\def\csname PY@tok@il\endcsname{\def\PY@tc##1{\textcolor[rgb]{0.40,0.40,0.40}{##1}}}
\expandafter\def\csname PY@tok@cs\endcsname{\let\PY@it=\textit\def\PY@tc##1{\textcolor[rgb]{0.25,0.50,0.50}{##1}}}
\expandafter\def\csname PY@tok@cp\endcsname{\def\PY@tc##1{\textcolor[rgb]{0.74,0.48,0.00}{##1}}}
\expandafter\def\csname PY@tok@gi\endcsname{\def\PY@tc##1{\textcolor[rgb]{0.00,0.63,0.00}{##1}}}
\expandafter\def\csname PY@tok@gh\endcsname{\let\PY@bf=\textbf\def\PY@tc##1{\textcolor[rgb]{0.00,0.00,0.50}{##1}}}
\expandafter\def\csname PY@tok@ni\endcsname{\let\PY@bf=\textbf\def\PY@tc##1{\textcolor[rgb]{0.60,0.60,0.60}{##1}}}
\expandafter\def\csname PY@tok@nl\endcsname{\def\PY@tc##1{\textcolor[rgb]{0.63,0.63,0.00}{##1}}}
\expandafter\def\csname PY@tok@nn\endcsname{\let\PY@bf=\textbf\def\PY@tc##1{\textcolor[rgb]{0.00,0.00,1.00}{##1}}}
\expandafter\def\csname PY@tok@no\endcsname{\def\PY@tc##1{\textcolor[rgb]{0.53,0.00,0.00}{##1}}}
\expandafter\def\csname PY@tok@na\endcsname{\def\PY@tc##1{\textcolor[rgb]{0.49,0.56,0.16}{##1}}}
\expandafter\def\csname PY@tok@nb\endcsname{\def\PY@tc##1{\textcolor[rgb]{0.00,0.50,0.00}{##1}}}
\expandafter\def\csname PY@tok@nc\endcsname{\let\PY@bf=\textbf\def\PY@tc##1{\textcolor[rgb]{0.00,0.00,1.00}{##1}}}
\expandafter\def\csname PY@tok@nd\endcsname{\def\PY@tc##1{\textcolor[rgb]{0.67,0.13,1.00}{##1}}}
\expandafter\def\csname PY@tok@ne\endcsname{\let\PY@bf=\textbf\def\PY@tc##1{\textcolor[rgb]{0.82,0.25,0.23}{##1}}}
\expandafter\def\csname PY@tok@nf\endcsname{\def\PY@tc##1{\textcolor[rgb]{0.00,0.00,1.00}{##1}}}
\expandafter\def\csname PY@tok@si\endcsname{\let\PY@bf=\textbf\def\PY@tc##1{\textcolor[rgb]{0.73,0.40,0.53}{##1}}}
\expandafter\def\csname PY@tok@s2\endcsname{\def\PY@tc##1{\textcolor[rgb]{0.73,0.13,0.13}{##1}}}
\expandafter\def\csname PY@tok@vi\endcsname{\def\PY@tc##1{\textcolor[rgb]{0.10,0.09,0.49}{##1}}}
\expandafter\def\csname PY@tok@nt\endcsname{\let\PY@bf=\textbf\def\PY@tc##1{\textcolor[rgb]{0.00,0.50,0.00}{##1}}}
\expandafter\def\csname PY@tok@nv\endcsname{\def\PY@tc##1{\textcolor[rgb]{0.10,0.09,0.49}{##1}}}
\expandafter\def\csname PY@tok@s1\endcsname{\def\PY@tc##1{\textcolor[rgb]{0.73,0.13,0.13}{##1}}}
\expandafter\def\csname PY@tok@sh\endcsname{\def\PY@tc##1{\textcolor[rgb]{0.73,0.13,0.13}{##1}}}
\expandafter\def\csname PY@tok@sc\endcsname{\def\PY@tc##1{\textcolor[rgb]{0.73,0.13,0.13}{##1}}}
\expandafter\def\csname PY@tok@sx\endcsname{\def\PY@tc##1{\textcolor[rgb]{0.00,0.50,0.00}{##1}}}
\expandafter\def\csname PY@tok@bp\endcsname{\def\PY@tc##1{\textcolor[rgb]{0.00,0.50,0.00}{##1}}}
\expandafter\def\csname PY@tok@c1\endcsname{\let\PY@it=\textit\def\PY@tc##1{\textcolor[rgb]{0.25,0.50,0.50}{##1}}}
\expandafter\def\csname PY@tok@kc\endcsname{\let\PY@bf=\textbf\def\PY@tc##1{\textcolor[rgb]{0.00,0.50,0.00}{##1}}}
\expandafter\def\csname PY@tok@c\endcsname{\let\PY@it=\textit\def\PY@tc##1{\textcolor[rgb]{0.25,0.50,0.50}{##1}}}
\expandafter\def\csname PY@tok@mf\endcsname{\def\PY@tc##1{\textcolor[rgb]{0.40,0.40,0.40}{##1}}}
\expandafter\def\csname PY@tok@err\endcsname{\def\PY@bc##1{\setlength{\fboxsep}{0pt}\fcolorbox[rgb]{1.00,0.00,0.00}{1,1,1}{\strut ##1}}}
\expandafter\def\csname PY@tok@kd\endcsname{\let\PY@bf=\textbf\def\PY@tc##1{\textcolor[rgb]{0.00,0.50,0.00}{##1}}}
\expandafter\def\csname PY@tok@ss\endcsname{\def\PY@tc##1{\textcolor[rgb]{0.10,0.09,0.49}{##1}}}
\expandafter\def\csname PY@tok@sr\endcsname{\def\PY@tc##1{\textcolor[rgb]{0.73,0.40,0.53}{##1}}}
\expandafter\def\csname PY@tok@mo\endcsname{\def\PY@tc##1{\textcolor[rgb]{0.40,0.40,0.40}{##1}}}
\expandafter\def\csname PY@tok@kn\endcsname{\let\PY@bf=\textbf\def\PY@tc##1{\textcolor[rgb]{0.00,0.50,0.00}{##1}}}
\expandafter\def\csname PY@tok@mi\endcsname{\def\PY@tc##1{\textcolor[rgb]{0.40,0.40,0.40}{##1}}}
\expandafter\def\csname PY@tok@gp\endcsname{\let\PY@bf=\textbf\def\PY@tc##1{\textcolor[rgb]{0.00,0.00,0.50}{##1}}}
\expandafter\def\csname PY@tok@o\endcsname{\def\PY@tc##1{\textcolor[rgb]{0.40,0.40,0.40}{##1}}}
\expandafter\def\csname PY@tok@kr\endcsname{\let\PY@bf=\textbf\def\PY@tc##1{\textcolor[rgb]{0.00,0.50,0.00}{##1}}}
\expandafter\def\csname PY@tok@s\endcsname{\def\PY@tc##1{\textcolor[rgb]{0.73,0.13,0.13}{##1}}}
\expandafter\def\csname PY@tok@kp\endcsname{\def\PY@tc##1{\textcolor[rgb]{0.00,0.50,0.00}{##1}}}
\expandafter\def\csname PY@tok@w\endcsname{\def\PY@tc##1{\textcolor[rgb]{0.73,0.73,0.73}{##1}}}
\expandafter\def\csname PY@tok@kt\endcsname{\def\PY@tc##1{\textcolor[rgb]{0.69,0.00,0.25}{##1}}}
\expandafter\def\csname PY@tok@ow\endcsname{\let\PY@bf=\textbf\def\PY@tc##1{\textcolor[rgb]{0.67,0.13,1.00}{##1}}}
\expandafter\def\csname PY@tok@sb\endcsname{\def\PY@tc##1{\textcolor[rgb]{0.73,0.13,0.13}{##1}}}
\expandafter\def\csname PY@tok@k\endcsname{\let\PY@bf=\textbf\def\PY@tc##1{\textcolor[rgb]{0.00,0.50,0.00}{##1}}}
\expandafter\def\csname PY@tok@se\endcsname{\let\PY@bf=\textbf\def\PY@tc##1{\textcolor[rgb]{0.73,0.40,0.13}{##1}}}
\expandafter\def\csname PY@tok@sd\endcsname{\let\PY@it=\textit\def\PY@tc##1{\textcolor[rgb]{0.73,0.13,0.13}{##1}}}

\def\PYZbs{\char`\\}
\def\PYZus{\char`\_}
\def\PYZob{\char`\{}
\def\PYZcb{\char`\}}
\def\PYZca{\char`\^}
\def\PYZam{\char`\&}
\def\PYZlt{\char`\<}
\def\PYZgt{\char`\>}
\def\PYZsh{\char`\#}
\def\PYZpc{\char`\%}
\def\PYZdl{\char`\$}
\def\PYZti{\char`\~}
% for compatibility with earlier versions
\def\PYZat{@}
\def\PYZlb{[}
\def\PYZrb{]}
\makeatother


\hypersetup {
        colorlinks = true,
        urlcolor = black,
        linkcolor = black,
        pdfauthor = {\nameD\nameH\nameB},
        pdfkeywords = {},
        pdfsubject = {},
        pdfpagemode = UseNone
}

\titleformat{\section}
	{\normalfont\fontsize{15}{10}\bfseries}{\thesection}{1em}{}
\titleformat{\subsection}
	{\normalfont\fontsize{12}{15}\bfseries}{\thesubsection}{1em}{}
\titleformat{\subsubsection}
	{\normalfont\fontsize{12}{15}\bfseries}{\thesubsubsection}{1em}{}

\begin{document}

\title{\vspace{20em}Technology Review \\{\vspace{-1ex}\huge UniversityGear} \\
{\large \today}}
\author{\vspace{10ex}Devin Foulger \\{\vspace{-1ex}Hector Trujillo}
\\{\vspace{-1ex}Bryan Liauw}}

\begin{titlepage}

\maketitle
\thispagestyle{empty}

\end{titlepage}

\tableofcontents

\section{Introduction}


\section{Technologies}
\subsection{Technology: Checkout}
Checking out is a feature that is essential to any application or website where
you purchase some item. This means that our application will need to allow the
user to check out the items in some fashion.

\subsubsection{Options}
\subsubsection*{Option 1: Checkout with sign-in}
Allowing the user to sign in to checkout items gives us the benefit of allowing
members to purchase from our application. In order to implement this feature, we 
would have to use eBay's Order API. This API would allow for the user to
purchase items using their existing eBay and PayPal accounts. This also allows
the user to view the shipping address provided and any fulfillment information.

\subsubsection*{Option 2: Checkout as guest user}
Allowing the user to checkout items through a guest check out would allow any
user to purchase from our application. In order to implement guest checkout, we
would need to use eBay's Order API with GUEST CHECKOUT SESSION. This is
different than using the Order API normally. With guest checkout, the user is
able to see that item that they have purchased and the status of the payment and
 order.

\subsubsection*{Option 3: Checkout is either guest user or with sign-in}
Allowing the user with the option to either sign-in or use guest checkout gives
us the option to offer which ever method makes the user most comfortable when
checking out. This would be that we would have to use eBay's Order API to the
fullest by implementing both sign-in and guest checkout. This means that the
user would be able to use PayPal, if they have an account, or checkout via
credit card.

\subsubsection{Use in design}
This is a fundamental use in our design. Without this feature, or application
would simply be a searching application. The user would only be able to search
for specific items that exist within eBay's stores, but not actually purchase
them.

\subsubsection{Cost, availability, speed, security}
\subsubsection*{Option 1: Checkout with sign-in}
The cost of checkout with sign-in is the development time it would take to
implement. This feature is tricky and would complicates the use of eBay's APIs.
However, the availability is limited to those that have an eBay account. The
speed is also largely dependent on internet connectivity and the status of
eBay's servers. Finally, the security is very tight. This is because the user
must authenticate before making a purchase. On top of that, all information that 
is used is sent via HTTPS.

\subsubsection*{Option 2: Checkout as guest user}
The cost of guest checkout is relatively low. It would take the least amount of
development time and resources. The availability is also very high allowing us
to reach anyone who would like to use our app. Again, the speed is dependent on
internet connectivity and the status of eBay's servers. However, the security
would be great. None of the user's information should be saved to the device.
That includes things like credit card information and shipping address. Also,
any information that is used is sent via HTTPS requests.

\subsubsection*{Option 3: Checkout is either guest user or with sign-in}
The cost of the guest checkout coupled with sign-in is very high. It would
require a significant amount of development time to complete. However, the
availability would be at its greatest because it would allow for any and all
users to purchase items. The speed of the checkout would, again, depend on the
status of eBay's servers and internet connectivity. The security is the same as
the above options.

\begin{table}[!h]
\centering
\caption{Comparison Table of Options for Checkout}
\label{Comparison Table of Options for Checkout}
\begin{tabularx}{\textwidth}{|l|l|l|l|X|}
\hline
\textbf{} & \textbf{Time Cost} & \textbf{Availability} & \textbf{Speed} &
\textbf{Security} \\ \hline
\textbf{Option 1} & High & To only members of eBay & Fast & Best, because it
offers an extra layer of authentication \\ \hline
\textbf{Option 2} & Low & To everyone as a guest & Best & Great \\ \hline
\textbf{Option 3} & Very high & To members of eBay and guests & This could
either be best or fast & This could be best or great \\ \hline
\end{tabularx}
\end{table}

\subsubsection{Evaluation}
Each option has their advantages and disadvantages. If we were to allow users to
 checkout only using sign-in, then we might only be targeting a smaller audience.
 The complexity for implementing such a feature is much harder than that of 
guest checkout as well. Using guest checkout would allow for all users to use our
application with no complicated sign in. It would also mean that users would not 
have to create an account to user our application. If we were to implement both
sign-in and guest checkout, the complexity would be too great. Given the small
amount of development time it might be hard to implement both features. 
However,this option would give us the ability to reach every user possible.

\subsubsection{Best Choice}
The best choice for our application is implementing a guest checkout. This is
because with a short development time, guest checkout would be the fastest to
implement. This would also allow any user to purchase from our application,
which means we would not be alienating someone who does not have an eBay
account.

\subsection{Technology: Checkout UI}
Checking out needs to have a UI that is easy to use. If we only stub out a UI,
it might be complicated, or worse, ugly. The UI needs to be attractive and
functional.

\subsubsection{Options}
\subsubsection*{Option 1: Layout Editor (Native Android Studio)}
The native android layout editor doesn't work very well for prototyping.
However, it allows for easy UI creation while being able to map the buttons to
correct functions. This tool doesn't offer much in terms of quick design, but it
offers a lot of functionality. It uses XML files to create the UI. It does allow
you to alter any aspect of your UI, which is very good as well. It also allows
you to create your own UI objects very easily.

\subsubsection*{Option 2: Indigo Studio}
Indigo Studio is a wire framing tool. With this tool I would be able to create
great looking wireframes. This doesn't allow for much in terms of actual
development though. It is just a tool for quick prototyping and design. However,
it does offer a lot in terms of customization and they also offer a lot of
images for buttons, list views, and things of that sort.

\subsubsection*{Option 3: Just In Mind}
Just In Mind is a comprehensive prototyping tool for many types of development,
including Android development. This tool offers a lot in terms of development.
There are many UI libraries for Android. This tool would also allow me to use
templates that they have already built. On top of that, any of the designs or
UIs I make can be transferred into already existing projects. This tool also
offers integration with other tools as well, like Photoshop. It also features
interactive images and animations for use in your projects.

\subsubsection{Use in design}
Having a functional and beautiful looking UI keeps users attracted to your app.
So, having a nice UI is a must in this project. The user will also constantly be
interacting with the application.

\subsubsection{Cost}
These tools are simply for creating clean and responsive UIs. That being said,
the native Android Studio Layout Editor is free. Indigo Studio costs 25 dollars
a month and Just In Mind costs 19 dollars a month.

\begin{table}[h]
\centering
\caption{Comparison Table of Options for Checkout UI}
\label{my-label}
\begin{tabularx}{\textwidth}{|l|l|X|}
\hline
\textbf{}         & \textbf{Cost}                & \textbf{Learning Curve} \\ \hline
\textbf{Option 1} & Free                                                & There
 is somewhat of a learning curve as it requires you to also develop at the same time 
 \\ \hline
\textbf{Option 2} & Free with limited features, then 25 dollars a month & Low                                                                                       
 \\ \hline
\textbf{Option 3} & Free with limited features, then 19 dollars a month & Low                                                               
 \\ \hline
\end{tabularx}
\end{table}

\subsubsection{Evaluation}
Each tool has clear disadvantages and advantages when compared to each other.
Just In Mind and Indigo Studio have relatively similar functions. However, Just
In Mind offers a bit more such as UI libraries that are compatible with Android
Studio. Indigo Studio does allow for faster creation of wireframes when compared
to Just In Mind. These two tools can be much better than the native layout
editor in Android Studio. The editor layout also offers its own features as
well. It gives developers the ability to customizing anything they like about
their UI.

\subsubsection{Best Choice}
The best choice for this project would be to use the native Android Studio
Layout Editor. Although the other tools offer a lot more in terms of
prototyping, they have a cost. If you don't pay for the tools, you don't get
everything that is included. However, Android Studio does offer the ultimate
tool for any type of customization you would like to make. You have full control
over what you are trying to do. You also don't have to worry about integration
with other tools, because it is built in.

\subsection{Technology: Getting data for Single Item View}
The data that is received from eBay's APIs comes in the form of an JSON file.
This JSON data needs to be parsed correctly so that the information is portrayed
to the user in a meaningful manner. Each item needs to correctly be displayed to
the user so that they can view the description or images of a particular item.

\subsubsection{Options}
\subsubsection*{Option 1: }
GSON allows for Java objects to be converted into JSON objects, and vice versa.
This is a tool that is not native to Android Studio, but offers a lot in terms
of use. It has many methods that allow for array creation which holds JSON
information. These arrays can then be used to fill the UI with relevant
information for the user to view.

\subsubsection*{Option 2: }
JsonReader is a light weight API that would allow me to easily read JSON files.
It allows me to create JSON objects and arrays. The arrays would hold
information that could easily be presented to the user.

\subsubsection*{Option 3: }
JSON Parser is native to Android Studio and is very comprehensive. It would
allow me to create objects using JSON files. I would have the ability to create
JSON arrays, object, and key-value pairs using JSON Parser. It includes many
methods for easy creation of JSON objects and parsing.

\subsubsection{Use in design}
A JSON parser is absolutely a must if we want to make sense of our data. This is
because eBay's APIs return the data in JSON format. In order to display the
information correctly to a user, we have to use a JSON parser.

\subsubsection{Cost and speed}
All three APIs do the same thing in the same fashion. They all create readable
JSON arrays that hold information to be used by the developer.

\subsubsection{Comparisons}
All three APIs do the same thing in the same fashion. They all create readable
JSON arrays that hold information to be used by the developer.

\subsubsection{Evaluation}
It is important to note that theses APIs perform almost identically. On top of
that, they all produce the same output. There is no clear advantage or
disadvantages over the others. The only disadvantage to the JsonReader, is that
you don't have the ability to create new JSON objects.

\subsubsection{Best Choice}
The best choice to use for our application would be the built in Android parser,
JSON Parser. This is because all the parsers perform similarly, but with the
built in parser, I get to avoid the headache of importing a new one. It also has
the ability to create new JSON objects if that is needed.

\section{Conclusion}


\end{document}