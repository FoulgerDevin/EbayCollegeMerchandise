\documentclass[journal,compsoc, 10pt, draftclsnofoot, onecolumn]{IEEEtran}

\usepackage{graphicx}
\usepackage{amssymb}
\usepackage{amsmath}
\usepackage{amsthm}
\usepackage{tabularx}
\usepackage{graphicx}

\newcommand{\subparagraph}{}
\usepackage{titlesec}

\usepackage{alltt}
\usepackage{float}
\usepackage{color}
\usepackage{url}

\usepackage{balance}
\usepackage[TABBOTCAP, tight]{subfigure}
\usepackage{enumitem}
\usepackage{pstricks, pst-node}

\usepackage{cite}
\usepackage{listings}

\usepackage[margin=0.75in]{geometry}
\geometry{textheight=8.5in, textwidth=6in}

\renewcommand{\familydefault}{\sfdefault}


\newlength\tindent
\setlength{\tindent}{\parindent}
\setlength{\parindent}{0pt}
\renewcommand{\indent}{\hspace*{\tindent}}

\newcommand{\cred}[1]{{\color{red}#1}}
\newcommand{\cblue}[1]{{\color{blue}#1}}

\newcommand{\namesigdate}[2][6cm]{%
  \begin{tabular}{@{}p{#1}@{}}
    #2 \\[0.5\normalbaselineskip] \hrule \\[0pt]
    {\small \textit{Signature}} \\[0.5\normalbaselineskip] \hrule \\[0pt]
    {\small \textit{Date}}
  \end{tabular}
}


\usepackage{hyperref}
\usepackage{geometry}

\lstset{
language=C,
basicstyle=\ttfamily,
commentstyle=\color{blue},
keywordstyle=\color{green},
numberstyle=\color{red},
stringstyle=\color{orange}
}

\def\nameD{Devin Foulger}
\def\nameH{Hector Trujillo}
\def\nameB{Bryan Liauw}

\usepackage{fancyvrb}
\usepackage{color}
\usepackage[latin1]{inputenc}


\makeatletter
\def\PY@reset{\let\PY@it=\relax \let\PY@bf=\relax%
    \let\PY@ul=\relax \let\PY@tc=\relax%
    \let\PY@bc=\relax \let\PY@ff=\relax}
\def\PY@tok#1{\csname PY@tok@#1\endcsname}
\def\PY@toks#1+{\ifx\relax#1\empty\else%
    \PY@tok{#1}\expandafter\PY@toks\fi}
\def\PY@do#1{\PY@bc{\PY@tc{\PY@ul{%
    \PY@it{\PY@bf{\PY@ff{#1}}}}}}}
\def\PY#1#2{\PY@reset\PY@toks#1+\relax+\PY@do{#2}}

\expandafter\def\csname PY@tok@gd\endcsname{\def\PY@tc##1{\textcolor[rgb]{0.63,0.00,0.00}{##1}}}
\expandafter\def\csname PY@tok@gu\endcsname{\let\PY@bf=\textbf\def\PY@tc##1{\textcolor[rgb]{0.50,0.00,0.50}{##1}}}
\expandafter\def\csname PY@tok@gt\endcsname{\def\PY@tc##1{\textcolor[rgb]{0.00,0.25,0.82}{##1}}}
\expandafter\def\csname PY@tok@gs\endcsname{\let\PY@bf=\textbf}
\expandafter\def\csname PY@tok@gr\endcsname{\def\PY@tc##1{\textcolor[rgb]{1.00,0.00,0.00}{##1}}}
\expandafter\def\csname PY@tok@cm\endcsname{\let\PY@it=\textit\def\PY@tc##1{\textcolor[rgb]{0.25,0.50,0.50}{##1}}}
\expandafter\def\csname PY@tok@vg\endcsname{\def\PY@tc##1{\textcolor[rgb]{0.10,0.09,0.49}{##1}}}
\expandafter\def\csname PY@tok@m\endcsname{\def\PY@tc##1{\textcolor[rgb]{0.40,0.40,0.40}{##1}}}
\expandafter\def\csname PY@tok@mh\endcsname{\def\PY@tc##1{\textcolor[rgb]{0.40,0.40,0.40}{##1}}}
\expandafter\def\csname PY@tok@go\endcsname{\def\PY@tc##1{\textcolor[rgb]{0.50,0.50,0.50}{##1}}}
\expandafter\def\csname PY@tok@ge\endcsname{\let\PY@it=\textit}
\expandafter\def\csname PY@tok@vc\endcsname{\def\PY@tc##1{\textcolor[rgb]{0.10,0.09,0.49}{##1}}}
\expandafter\def\csname PY@tok@il\endcsname{\def\PY@tc##1{\textcolor[rgb]{0.40,0.40,0.40}{##1}}}
\expandafter\def\csname PY@tok@cs\endcsname{\let\PY@it=\textit\def\PY@tc##1{\textcolor[rgb]{0.25,0.50,0.50}{##1}}}
\expandafter\def\csname PY@tok@cp\endcsname{\def\PY@tc##1{\textcolor[rgb]{0.74,0.48,0.00}{##1}}}
\expandafter\def\csname PY@tok@gi\endcsname{\def\PY@tc##1{\textcolor[rgb]{0.00,0.63,0.00}{##1}}}
\expandafter\def\csname PY@tok@gh\endcsname{\let\PY@bf=\textbf\def\PY@tc##1{\textcolor[rgb]{0.00,0.00,0.50}{##1}}}
\expandafter\def\csname PY@tok@ni\endcsname{\let\PY@bf=\textbf\def\PY@tc##1{\textcolor[rgb]{0.60,0.60,0.60}{##1}}}
\expandafter\def\csname PY@tok@nl\endcsname{\def\PY@tc##1{\textcolor[rgb]{0.63,0.63,0.00}{##1}}}
\expandafter\def\csname PY@tok@nn\endcsname{\let\PY@bf=\textbf\def\PY@tc##1{\textcolor[rgb]{0.00,0.00,1.00}{##1}}}
\expandafter\def\csname PY@tok@no\endcsname{\def\PY@tc##1{\textcolor[rgb]{0.53,0.00,0.00}{##1}}}
\expandafter\def\csname PY@tok@na\endcsname{\def\PY@tc##1{\textcolor[rgb]{0.49,0.56,0.16}{##1}}}
\expandafter\def\csname PY@tok@nb\endcsname{\def\PY@tc##1{\textcolor[rgb]{0.00,0.50,0.00}{##1}}}
\expandafter\def\csname PY@tok@nc\endcsname{\let\PY@bf=\textbf\def\PY@tc##1{\textcolor[rgb]{0.00,0.00,1.00}{##1}}}
\expandafter\def\csname PY@tok@nd\endcsname{\def\PY@tc##1{\textcolor[rgb]{0.67,0.13,1.00}{##1}}}
\expandafter\def\csname PY@tok@ne\endcsname{\let\PY@bf=\textbf\def\PY@tc##1{\textcolor[rgb]{0.82,0.25,0.23}{##1}}}
\expandafter\def\csname PY@tok@nf\endcsname{\def\PY@tc##1{\textcolor[rgb]{0.00,0.00,1.00}{##1}}}
\expandafter\def\csname PY@tok@si\endcsname{\let\PY@bf=\textbf\def\PY@tc##1{\textcolor[rgb]{0.73,0.40,0.53}{##1}}}
\expandafter\def\csname PY@tok@s2\endcsname{\def\PY@tc##1{\textcolor[rgb]{0.73,0.13,0.13}{##1}}}
\expandafter\def\csname PY@tok@vi\endcsname{\def\PY@tc##1{\textcolor[rgb]{0.10,0.09,0.49}{##1}}}
\expandafter\def\csname PY@tok@nt\endcsname{\let\PY@bf=\textbf\def\PY@tc##1{\textcolor[rgb]{0.00,0.50,0.00}{##1}}}
\expandafter\def\csname PY@tok@nv\endcsname{\def\PY@tc##1{\textcolor[rgb]{0.10,0.09,0.49}{##1}}}
\expandafter\def\csname PY@tok@s1\endcsname{\def\PY@tc##1{\textcolor[rgb]{0.73,0.13,0.13}{##1}}}
\expandafter\def\csname PY@tok@sh\endcsname{\def\PY@tc##1{\textcolor[rgb]{0.73,0.13,0.13}{##1}}}
\expandafter\def\csname PY@tok@sc\endcsname{\def\PY@tc##1{\textcolor[rgb]{0.73,0.13,0.13}{##1}}}
\expandafter\def\csname PY@tok@sx\endcsname{\def\PY@tc##1{\textcolor[rgb]{0.00,0.50,0.00}{##1}}}
\expandafter\def\csname PY@tok@bp\endcsname{\def\PY@tc##1{\textcolor[rgb]{0.00,0.50,0.00}{##1}}}
\expandafter\def\csname PY@tok@c1\endcsname{\let\PY@it=\textit\def\PY@tc##1{\textcolor[rgb]{0.25,0.50,0.50}{##1}}}
\expandafter\def\csname PY@tok@kc\endcsname{\let\PY@bf=\textbf\def\PY@tc##1{\textcolor[rgb]{0.00,0.50,0.00}{##1}}}
\expandafter\def\csname PY@tok@c\endcsname{\let\PY@it=\textit\def\PY@tc##1{\textcolor[rgb]{0.25,0.50,0.50}{##1}}}
\expandafter\def\csname PY@tok@mf\endcsname{\def\PY@tc##1{\textcolor[rgb]{0.40,0.40,0.40}{##1}}}
\expandafter\def\csname PY@tok@err\endcsname{\def\PY@bc##1{\setlength{\fboxsep}{0pt}\fcolorbox[rgb]{1.00,0.00,0.00}{1,1,1}{\strut ##1}}}
\expandafter\def\csname PY@tok@kd\endcsname{\let\PY@bf=\textbf\def\PY@tc##1{\textcolor[rgb]{0.00,0.50,0.00}{##1}}}
\expandafter\def\csname PY@tok@ss\endcsname{\def\PY@tc##1{\textcolor[rgb]{0.10,0.09,0.49}{##1}}}
\expandafter\def\csname PY@tok@sr\endcsname{\def\PY@tc##1{\textcolor[rgb]{0.73,0.40,0.53}{##1}}}
\expandafter\def\csname PY@tok@mo\endcsname{\def\PY@tc##1{\textcolor[rgb]{0.40,0.40,0.40}{##1}}}
\expandafter\def\csname PY@tok@kn\endcsname{\let\PY@bf=\textbf\def\PY@tc##1{\textcolor[rgb]{0.00,0.50,0.00}{##1}}}
\expandafter\def\csname PY@tok@mi\endcsname{\def\PY@tc##1{\textcolor[rgb]{0.40,0.40,0.40}{##1}}}
\expandafter\def\csname PY@tok@gp\endcsname{\let\PY@bf=\textbf\def\PY@tc##1{\textcolor[rgb]{0.00,0.00,0.50}{##1}}}
\expandafter\def\csname PY@tok@o\endcsname{\def\PY@tc##1{\textcolor[rgb]{0.40,0.40,0.40}{##1}}}
\expandafter\def\csname PY@tok@kr\endcsname{\let\PY@bf=\textbf\def\PY@tc##1{\textcolor[rgb]{0.00,0.50,0.00}{##1}}}
\expandafter\def\csname PY@tok@s\endcsname{\def\PY@tc##1{\textcolor[rgb]{0.73,0.13,0.13}{##1}}}
\expandafter\def\csname PY@tok@kp\endcsname{\def\PY@tc##1{\textcolor[rgb]{0.00,0.50,0.00}{##1}}}
\expandafter\def\csname PY@tok@w\endcsname{\def\PY@tc##1{\textcolor[rgb]{0.73,0.73,0.73}{##1}}}
\expandafter\def\csname PY@tok@kt\endcsname{\def\PY@tc##1{\textcolor[rgb]{0.69,0.00,0.25}{##1}}}
\expandafter\def\csname PY@tok@ow\endcsname{\let\PY@bf=\textbf\def\PY@tc##1{\textcolor[rgb]{0.67,0.13,1.00}{##1}}}
\expandafter\def\csname PY@tok@sb\endcsname{\def\PY@tc##1{\textcolor[rgb]{0.73,0.13,0.13}{##1}}}
\expandafter\def\csname PY@tok@k\endcsname{\let\PY@bf=\textbf\def\PY@tc##1{\textcolor[rgb]{0.00,0.50,0.00}{##1}}}
\expandafter\def\csname PY@tok@se\endcsname{\let\PY@bf=\textbf\def\PY@tc##1{\textcolor[rgb]{0.73,0.40,0.13}{##1}}}
\expandafter\def\csname PY@tok@sd\endcsname{\let\PY@it=\textit\def\PY@tc##1{\textcolor[rgb]{0.73,0.13,0.13}{##1}}}

\def\PYZbs{\char`\\}
\def\PYZus{\char`\_}
\def\PYZob{\char`\{}
\def\PYZcb{\char`\}}
\def\PYZca{\char`\^}
\def\PYZam{\char`\&}
\def\PYZlt{\char`\<}
\def\PYZgt{\char`\>}
\def\PYZsh{\char`\#}
\def\PYZpc{\char`\%}
\def\PYZdl{\char`\$}
\def\PYZti{\char`\~}
% for compatibility with earlier versions
\def\PYZat{@}
\def\PYZlb{[}
\def\PYZrb{]}
\makeatother


\hypersetup {
        colorlinks = true,
        urlcolor = black,
        linkcolor = black,
        pdfauthor = {\nameD\nameH\nameB},
        pdfkeywords = {},
        pdfsubject = {},
        pdfpagemode = UseNone
}

\titleformat{\section}
	{\normalfont\fontsize{15}{10}\bfseries}{\thesection}{1em}{}
\titleformat{\subsection}
	{\normalfont\fontsize{12}{15}\bfseries}{\thesubsection}{1em}{}
\titleformat{\subsubsection}
	{\normalfont\fontsize{12}{15}\bfseries}{\thesubsubsection}{1em}{}

\begin{document}

\title{\vspace{20em}Technology Review \\{\vspace{-1ex}\huge UniversityGear} \\
{\large \today}}
\author{\vspace{10ex}Devin Foulger \\{\vspace{-1ex}Hector Trujillo}
\\{\vspace{-1ex}Bryan Liauw}}

\begin{titlepage}

\maketitle
\thispagestyle{empty}

\end{titlepage}

\tableofcontents

\section{Introduction}


\section{Technologies}
\subsection{Technology: Checkout}
Checking out is a feature that is essential to any application or website where
you purchase some item. This means that our application will need to allow the
user to check out the items in some fashion.

\subsubsection{Options}
\subsubsection*{Option 1: Checkout with sign-in}
Allowing the user to sign in to checkout items gives us the benefit of allowing
members to purchase from our application. In order to implement this feature, we 
would have to use eBay's Order API. This API would allow for the user to
purchase items using their existing eBay and PayPal accounts. This also allows
the user to view the shipping address provided and any fulfillment information.

\subsubsection*{Option 2: Checkout as guest user}
Allowing the user to checkout items through a guest check out would allow any
user to purchase from our application. In order to implement guest checkout, we
would need to use eBay's Order API with GUEST CHECKOUT SESSION. This is
different than using the Order API normally. With guest checkout, the user is
able to see that item that they have purchased and the status of the payment and
 order.

\subsubsection*{Option 3: Checkout as either guest user or with sign-in}
Allowing the user with the option to either sign-in or use guest checkout gives
us the option to offer which ever method makes the user most comfortable when
checking out. This would be that we would have to use eBay's Order API to the
fullest by implementing both sign-in and guest checkout. This means that the
user would be able to use PayPal, if they have an account, or checkout via
credit card.

\subsubsection{Use in design}
This is a fundamental use in our design. Without this feature, or application
would simply be a searching application. The user would only be able to search
for specific items that exist within eBay's stores, but not actually purchase
them.

\subsubsection{Cost, availability, speed, security}
\subsubsection*{Option 1: Checkout with sign-in}
The cost of checkout with sign-in is the development time it would take to
implement. This feature is tricky and would complicates the use of eBay's APIs.
However, the availability is limited to those that have an eBay account. The
speed is also largely dependent on internet connectivity and the status of
eBay's servers. Finally, the security is very tight. This is because the user
must authenticate before making a purchase. On top of that, all information that 
is used is sent via HTTPS.

\subsubsection*{Option 2: Checkout as guest user}
The cost of guest checkout is relatively low. It would take the least amount of
development time and resources. The availability is also very high allowing us
to reach anyone who would like to use our app. Again, the speed is dependent on
internet connectivity and the status of eBay's servers. However, the security
would be great. None of the user's information should be saved to the device.
That includes things like credit card information and shipping address. Also,
any information that is used is sent via HTTPS requests.

\subsubsection*{Option 3: Checkout is either guest user or with sign-in}
The cost of the guest checkout coupled with sign-in is very high. It would
require a significant amount of development time to complete. However, the
availability would be at its greatest because it would allow for any and all
users to purchase items. The speed of the checkout would, again, depend on the
status of eBay's servers and internet connectivity. The security is the same as
the above options.

\begin{table}[!h]
\centering
\caption{Comparison Table of Options for Checkout}
\label{Comparison Table of Options for Checkout}
\begin{tabularx}{\textwidth}{|l|l|l|l|X|}
\hline
\textbf{} & \textbf{Time Cost} & \textbf{Availability} & \textbf{Speed} &
\textbf{Security} \\ \hline
\textbf{Option 1} & High & To only members of eBay & Fast & Best, because it
offers an extra layer of authentication \\ \hline
\textbf{Option 2} & Low & To everyone as a guest & Best & Great \\ \hline
\textbf{Option 3} & Very high & To members of eBay and guests & This could
either be best or fast & This could be best or great \\ \hline
\end{tabularx}
\end{table}

\subsubsection{Evaluation}
Each option has their advantages and disadvantages. If we were to allow users to
 checkout only using sign-in, then we might only be targeting a smaller audience.
 The complexity for implementing such a feature is much harder than that of 
guest checkout as well. Using guest checkout would allow for all users to use our
application with no complicated sign in. It would also mean that users would not 
have to create an account to user our application. If we were to implement both
sign-in and guest checkout, the complexity would be too great. Given the small
amount of development time it might be hard to implement both features. 
However,this option would give us the ability to reach every user possible.

\subsubsection{Best Choice}
The best choice for our application is implementing a guest checkout. This is
because with a short development time, guest checkout would be the fastest to
implement. This would also allow any user to purchase from our application,
which means we would not be alienating someone who does not have an eBay
account.

\subsection{Technology: Checkout UI}
Checking out needs to have a UI that is easy to use. If we only stub out a UI,
it might be complicated, or worse, ugly. The UI needs to be attractive and
functional.

\subsubsection{Options}
\subsubsection*{Option 1: Layout Editor (Native Android Studio)}
The native android layout editor doesn't work very well for prototyping.
However, it allows for easy UI creation while being able to map the buttons to
correct functions. This tool doesn't offer much in terms of quick design, but it
offers a lot of functionality. It uses XML files to create the UI. It does allow
you to alter any aspect of your UI, which is very good as well. It also allows
you to create your own UI objects very easily.

\subsubsection*{Option 2: Indigo Studio}
Indigo Studio is a wire framing tool. With this tool I would be able to create
great looking wireframes. This doesn't allow for much in terms of actual
development though. It is just a tool for quick prototyping and design. However, 
it does offer a lot in terms of customization and they also offer a lot of
images for buttons, list views, and things of that sort.

\subsubsection*{Option 3: Just In Mind}
Just In Mind is a comprehensive prototyping tool for many types of development,
including Android development. This tool offers a lot in terms of development.
There are many UI libraries for Android. This tool would also allow me to use
templates that they have already built. On top of that, any of the designs or
UIs I make can be transferred into already existing projects. This tool also
offers integration with other tools as well, like Photoshop. It also features
interactive images and animations for use in your projects.

\subsubsection{Use in design}
Having a functional and beautiful looking UI keeps users attracted to your app.
So, having a nice UI is a must in this project. The user will also constantly be
interacting with the application.

\subsubsection{Cost}
These tools are simply for creating clean and responsive UIs. That being said,
the native Android Studio Layout Editor is free. Indigo Studio costs 25 dollars
a month and Just In Mind costs 19 dollars a month.

\begin{table}[h]
\centering
\caption{Comparison Table of Options for Checkout UI}
\label{my-label}
\begin{tabularx}{\textwidth}{|l|l|X|}
\hline
\textbf{}         & \textbf{Cost}                & \textbf{Learning Curve} \\ \hline
\textbf{Option 1} & Free                                                & There
is somewhat of a learning curve as it requires you to also develop at the same time \\ \hline
\textbf{Option 2} & Free with limited features, then 25 dollars a month & Low \\ \hline
\textbf{Option 3} & Free with limited features, then 19 dollars a month & Low \\ \hline
\end{tabularx}
\end{table}

\subsubsection{Evaluation}
Each tool has clear disadvantages and advantages when compared to each other.
Just In Mind and Indigo Studio have relatively similar functions. However, Just
In Mind offers a bit more such as UI libraries that are compatible with Android
Studio. Indigo Studio does allow for faster creation of wireframes when compared
to Just In Mind. These two tools can be much better than the native layout
editor in Android Studio. The editor layout also offers its own features as
well. It gives developers the ability to customizing anything they like about
their UI.

\subsubsection{Best Choice}
The best choice for this project would be to use the native Android Studio
Layout Editor. Although the other tools offer a lot more in terms of
prototyping, they have a cost. If you don't pay for the tools, you don't get
everything that is included. However, Android Studio does offer the ultimate
tool for any type of customization you would like to make. You have full control
over what you are trying to do. You also don't have to worry about integration
with other tools, because it is built in.

\subsection{Technology: Getting data for Single Item View}
The data that is received from eBay's APIs comes in the form of an JSON file.
This JSON data needs to be parsed correctly so that the information is portrayed
to the user in a meaningful manner. Each item needs to correctly be displayed to
the user so that they can view the description or images of a particular item.

\subsubsection{Options}
\subsubsection*{Option 1: }
GSON allows for Java objects to be converted into JSON objects, and vice versa.
This is a tool that is not native to Android Studio, but offers a lot in terms
of use. It has many methods that allow for array creation which holds JSON
information. These arrays can then be used to fill the UI with relevant
information for the user to view.

\subsubsection*{Option 2: }
JsonReader is a light weight API that would allow me to easily read JSON files.
It allows me to create JSON objects and arrays. The arrays would hold
information that could easily be presented to the user.

\subsubsection*{Option 3: }
JSON Parser is native to Android Studio and is very comprehensive. It would
allow me to create objects using JSON files. I would have the ability to create
JSON arrays, object, and key-value pairs using JSON Parser. It includes many
methods for easy creation of JSON objects and parsing.

\subsubsection{Use in design}
A JSON parser is absolutely a must if we want to make sense of our data. This is
because eBay's APIs return the data in JSON format. In order to display the
information correctly to a user, we have to use a JSON parser.

\subsubsection{Cost and speed}
All three APIs do the same thing in the same fashion. They all create readable
JSON arrays that hold information to be used by the developer.

\subsubsection{Comparisons}
All three APIs do the same thing in the same fashion. They all create readable
JSON arrays that hold information to be used by the developer.

\subsubsection{Evaluation}
It is important to note that theses APIs perform almost identically. On top of
that, they all produce the same output. There is no clear advantage or
disadvantages over the others. The only disadvantage to the JsonReader, is that
you don't have the ability to create new JSON objects.

\subsubsection{Best Choice}
The best choice to use for our application would be the built in Android parser,
JSON Parser. This is because all the parsers perform similarly, but with the
built in parser, I get to avoid the headache of importing a new one. It also has
the ability to create new JSON objects if that is needed.

\subsection{Technology: Correcting User's Search Keyword}
Sometimes, when a user wants to find an item, he/she will mistype what they want to
search. This is a more prevalent issue in mobile apps since the keyboard input is
much smaller in smart phones. If we do not address this issue, the return value from
the search will not be optimal. In order for our app to understand what the user
wants to find despite the misspelled keyword, we need a fuzzy search algorithm.
There is 3 fuzzy search algorithms that comes into consideration: Levenshtein
Distance, Hamming Distance and Metaphone.

\subsubsection{Options}
\subsubsection*{Option 1: Levenshtein Distance}
Levenshtein Distance calculates the similarity of two strings and assign a numerical
value that determine how different the two strings are. This algorithm, given a
string s and a target string t, will return the number of alterations to s so that it
will be the same as t. The algorithm will construct a matrix m which has the size
m[s.stringlength][t.stringlength]. The matrix will be built on the minimum from this
rule:

\begin{lstlisting}
	1. If s[n] == t[n] where n is != t.stringlength 
		m[n][n] = 0 
		else if s[n] != t[n] 
			m[n][n] = 1
	
	2. m[a][b] where a < s.stringlength and b < t.stringlength 
		=  m[a-1][b]+1 or m[a][b-1]+1 
	whichever is smaller
	
	3. m[a][b] where a < s.stringlength and b < t.stringlength 
		=  m[a-1][b-1]+1
\end{lstlisting}

\subsubsection*{Option 2: Hamming Distance}
Hamming Distance is another algorithm that compares the difference between two
strings and assign a numerical value on how close one string is from another.
However, unlike Levenshtein Distance, the Hamming assigns value similar to a binary.
This algorithm will create an array of integer with the size of the length of the
string. This array's value at a given position n is 1, if the character of the first
string and the second string at n is the same, or 0 otherwise. The array's value is
totaled up and it will give a value on how different one string is from another.

\subsubsection*{Option 3: Metaphone}
Metaphone, on the other hand, does not assign a numerical value to check the
difference. It mainly checks if some character is substituted with its homophone. For
example, the algorithm will consider K and C similar and so is "kall" and "call".

\subsubsection{Comparison}
The main criteria that we evaluate is the reliability of the algorithm to find what
the user might mean when they accidentally mistyped a word. Secondary criteria
include maintainability and speed.

\begin{table}[h]
	\centering
	\caption{Comparison Table for the Fuzzy Search Algorithms}
	\label{Comparison Table for the Fuzzy Search Algorithms}
	\begin{tabularx}{\textwidth}{|l|l|l|X|}
		\hline
\textbf{} & \textbf{Reliability} & \textbf{Maintainability} & \textbf{Speed}
		\\ \hline
		\textbf{Option 1} & Applicable to most cases & Easy to maintain & Slower
		\\ \hline
\textbf{Option 2} & Applicable to most cases except non-equal string & Easy to
maintain & Fast
		\\ \hline
\textbf{Option 3} & Not applicable in most cases & Harder to maintain & Fast		\\ \hline
	\end{tabularx}
\end{table}

\subsubsection{Best choice}
For our application, Levenshtein Distance is the best option. Metaphone does not take
into account if user's input is not homophone to what the user wants to find. This
might be a problem if user is searching for certain types of brand. Also, adding
language support in the future means that we need to change some parts of the code to
cater to the new language since there will be different pronunciation. We will need
to change how our algorithm detect homophone if it happens and it will take
unnecessary time. On the other hand, the Hamming Distance is a faster algorithm than
Levenshtein since we only need to construct one-dimensional array as compared to
Levenshtein's two. But, Hamming Distance cannot give an accurate number if the length
of string is not equal. This can happen in the application if the user accidentally
sends in an incomplete keyword or simply do not know the exact keyword he/she is
looking for. Despite the trade-off in speed, we believe that Levenshtein's
reliability is more suitable for our application in order for user to have an optimal
session.

\subsection{Technology: Search Item in eBay's Database}
User will be able to find certain items in the application. The user will be able to
enter keyword and the app will return items that matches, or approximately matches,
user's query. This search result will be used alongside filters in order to get the
exact item that user wants.
The 3 ways we can go around this search is: Binary Search, Linear Search, eBay's
Finding API

\subsubsection{Options}
\subsubsection*{Option 1: Binary Search}
Binary search involves dividing the array into multiple halves until a match is
found. If the item to be searched is larger than the middle value, then the top half
will be the half to be searched. Else, the lower half will be searched. This search
algorithm's complexity is log n since the array size is halved every iteration. It is
considerably fast, especially when used in a larger database. However, the main
drawback is that the list must be sorted for this to work. Without a sorted list,
dividing the halves would not help finding the item since the top half and lower half
might not be larger or smaller respectively.

\subsubsection*{Option 2: Linear Search}
Linear Search is one of the most basic search. This involves checking every item in
the list and find if it matches the item. This algorithm's complexity is n since it
will look for every item. Considering the complexity, the speed of the algorithm is
slow especially in larger files. While smaller database would mean unnoticeable
difference in speed, it is very unlikely that our application will utilize small
database. The advantage of linear search is that there is no need to manipulate the
list of data we are looking in. Since the search will iterate over every single item
and match it with the keyword, there is no need to sort like binary search.
Therefore, despite the trade-off in speed, linear search makes it up with
reliability.

\subsubsection*{Option 3: Ebay Finding API}
eBay has a released API that search their database. It has a lot of function that
helps finding based on the query. Furthermore, it has several functions that helps
narrow down user's selection through filters. Despite not being transparent in how
their search works, eBay does widen the way a query can be searched, such as by
product or by keywords. As a bonus. the eBay API helps in manipulating the way search
results are returned. Since we are going to use eBay's sell and buy API, this API
will integrate well with other APIs.

\subsubsection{Comparison}
The main criteria are speed and reliability. Some other consideration would be
integrating the result with other functions later.

\begin{table}[h]
	\centering
	\caption{Comparison Table for the Search Algorithms}
	\label{Comparison Table for the Search Algorithm}
	\begin{tabularx}{\textwidth}{|X|X|X|}
		\hline
		\textbf{}         & \textbf{Reliability}                & \textbf{Speed} 
		\\ \hline
		\textbf{Option 1} & Low & Fast
		\\ \hline
\textbf{Option 2} & High & Slow		\\ \hline
\textbf{Option 3} & High &Fast(?)		\\ \hline
	\end{tabularx}
\end{table}

\subsubsection{Best choice}
We believe that the best approach would be using the Finding API. Since we are using
eBay's database, we have no guarantee that it would be sorted, therefore making
binary search a risk. On the other hand, if it is sorted, using a linear search would
make it run slower. We believe that the Finding API would be optimized by eBay and
will find the item fast and accurately. Furthermore, giving a lot of options such as
filtering will help us in the future. The integration with other API that we will use
is also a convenience that we should take advantage of.
  
\subsection{Technology: Saving Search for Future Purposes}
Our application will store user's search history internally in their phone. This is
so that we can give an accurate recommendation of items based on user's past search
history. The search history would be saved in a .txt format in the application
folder. There is two ways we can go around this: we can save the item id and search
it in the database in order to understand what the user's preferred item is or we can
save the keywords and determine user's preference through those keywords.
There is 3 ways we can save this txt file: Internal storage, External Storage, or
Server

\subsubsection{Options}
\subsubsection*{Option 1: Internal}
Internal storage means that the application would save user's preference inside their
mobile phone. This means that the user has total security from other source. Android
makes sure that the file saved internally is, under normal circumstances, only
accessible by the application itself. Furthermore, since the files are saved
internally, the application's read and write speed is faster by a certain margin;
most likely to be noticeable in lower-end hardware. However, the internal storage of
most Android smart phone is rather limited compared to external ones, therefore,
saving a huge file might frustrate the user.

\subsubsection*{Option 2: External}
Relying on external Storage using SD card or other similar hardware means we are
going to operate under the assumption that the user has such external storage. While
it is not an uncommon situation, some user might neglect having an external storage,
so it is not quite ideal. However, using external storage means high load of data is
possible. We would not worry about utilizing too much user's data. Although its read
and write speed will be slower, but having an external storage means that it is
possible to offload some processes to the memory card instead. However, security can
be easily compromised if the external storage is detached and lost. The data can
easily be retrieved once someone manage to get into the external storage.

\subsubsection*{Option 3: Server}
Using server will require internet connection. Given our application rely heavily on
internet, we can safely assume that user will have some kind of connection that
allows us to utilize an online database. By saving it into the server we do not need
to worry about the data load, since it is going to be handled elsewhere. The
trade-off would be the speed to process things, which is highly dependent on the
internet connection the user has. This can range from being very fast to very slow
depending on where the user is geographically.

\subsubsection{Comparison}
The speed on which the file can be taken would be very important, because we might
need it for different purposes. If the speed is too slow, user might get the feeling
that the application is not responding fast enough, which is something we must avoid.
Secondary criteria would be security and storage space.

\begin{table}[h]
	\centering
	\caption{Comparison Table for the Storage Medium}
	\label{Comparison Table for the Storage Medium}
	\begin{tabularx}{\textwidth}{|X|X|X|X|}
		\hline
\textbf{}         & \textbf{Security}                & \textbf{Speed} &\textbf{Space}		\\ \hline
		\textbf{Option 1} & Very Secure & Depends on the phone & Limited
		\\ \hline
\textbf{Option 2} & Not secure & Depends on the phone but slower than internal
Storage & Depends on the hardware but generally larger than internal storage
		\\ \hline
\textbf{Option 3} & Mostly very secure &Depends on the phone and internet connection
& Very huge but comes with a cost.
		\\ \hline
	\end{tabularx}
\end{table}

\subsubsection{Best choice}
Our application will use the internal storage with the option to allow user to use
external storage. We believe that the speed on which the information is retrieved is
very important. The data we are going to put in is not going to be very large; if the
data gets too large, we will have to narrow it down by removing past searches to
indicate the user's changed preferences. Using a server database online would add a
different layer of complication and unnecessary because the data would be small in
size. Furthermore, we won't have any other operation on the data which makes it
unnecessary for it to be saved on a server. We respect the users and their security,
which is why we would try to use the most secure method possible. However, if the
user does not have enough space internally, we would give an option to trade that
security and speed for more space.

\subsection{Optional Task: Machine Learning to Study User's Preference}
Once we have user's history of searches, our application should make use of that by
constructing user's preferences based on the information it stores. Additionally, we
can also use data that might be provided from the Ebay server to add more data in
order to get higher accuracy. This preference will be used for multiple purposes,
which can range from showing items on the home screen of the application, to
suggesting the user what to buy at the buying page. We construct this preference
using machine learning.
The 3 approach we can use to determine user's preference is Bayesian Estimate,
Apriori Algorithm, FP-Growth

\subsubsection{Options}
\subsubsection*{Option 1: Bayesian}
Bayesian algorithm involves comparing whether one thing is going to happen following
a certain event or if another event is likelier. This is calculated by learning the
pattern to what is considered as true or false. In terms of this application, true
would be user''s preference. The algorithm counts the combination of prior events and
its results and construct a possibility based on a given event. Given high
possibility, we can assume that the user will want that item. This operation,
however, might be time-consuming because there is a lot of mathematical calculations
as well as receiving data.

\subsubsection*{Option 2: Apriori}
The Apriori algorithm calculates the most probable item or combination of items. The
algorithm involves counting the number of occurrences where each item is selected in
the data. Then, the number below a given threshold, which is usually half of the
number of data, is removed. Then each item is then paired together and we count the
number of occurrence where the pair is selected. Again, the number below the
threshold is removed. This keeps repeating until we decide to stop and we will get
the most probable item that user might want. Considering this, the running time might
be long because we need to iterate through the data every time we want a larger size.
Since we might also need a temporary storage for the lists, memory consumption might
cause lower-end smartphones to lag, something we want to avoid the user from
experiencing.

\subsubsection*{Option 3: FP-Growth}
Similar to Apriori algorithm, the algorithm will count the number of occurrences for
each item. However, instead of iterating and growing the items, FP-Growth will create
a tree based on the data given. Each item will be represented by a node. The node
will have a value on it which depends on how many occurrences. The node is going to
be connected based on the data; so given data set {1,2,3} and {1,5,6}, the tree will
branch out at item 1 and the value of the node at item 1 will be 2. The route with
the highest value will be the most probable outcome. This algorithm is significantly
faster than Apriori although it will not create as much possibilities.

\subsubsection{Comparison}
The speed and the accuracy of the algorithm. Since slow algorithm will look
unresponsive, we would very much like to avoid that situation.

\begin{table}[h]
	\centering
	\caption{Comparison Table for the Machine Learning Algorithms}
	\label{Comparison Table for the Machine Learning Algorithms}
	\begin{tabularx}{\textwidth}{|X|X|X|}
		\hline
		\textbf{}         & \textbf{Accuracy}                & \textbf{Speed} 
		\\ \hline
\textbf{Option 1} & Accurate and can give lots of probable items & Slow, since it
might need to run through prior data several times
		\\ \hline
\textbf{Option 2} & Accurate and can give a lot of probable items & Slow, exponential
depending on how many iteration is needed
		\\ \hline
\textbf{Option 3} & Accurate but limited items & Fast, only requires at most 2 passes		\\ \hline
	\end{tabularx}
\end{table}

\subsubsection{Best choice}
We believe that FP-Growth might be the best choice. We value speed in our
application. User will feel that the app has too many loading times or unresponsive
if the algorithm takes much time. Considering the file might be larger as user uses
more of it, it is not recommended to use exponentially long algorithm.

\subsection{Technology: Viewing Multiple Search Results}
After a user submits a search, the API could potentially return hundreds of results
that will need to be presented to the user in an organized manner.

\subsubsection{Options}
\subsubsection*{Option 1: Single item view with swiping}
We could present one item at a time which would allow us to present more details of
the item. To view more items, the user would swipe left or right to go through the 
rest of the items in the search results. This would also eliminate the need for the 
user to click on the item to view more details as the relevant details would already 
be presented. The user would have enough information to know if they want to 
purchase the item. The ``Buy'' button would be on this view to allow quick purchases. 
\subsubsection*{Option 2: Item list view with pages}

We could also present the items in a list view, that would display X number of 
results per page. The users would swipe up or down to browse through the different 
items presented. This view would provide minimal details such as a short item 
description, and a thumbnail image. There would be a ``Next'' and ``Prev'' buttons 
that would allow the user to browse through the rest of the results that are not 
already presented.

\subsubsection*{Option 3: Item list view without pages}
This would be like the previous, present the items in a list view. The difference 
would be that instead of browsing through several pages, once the user reaches the 
end of the search results that are present more results would load, if they're 
available. Otherwise a message would be displayed at the bottom of the page saying 
``There are no more search results''.

\subsubsection{Use in design}
This is an important use as without a proper presentation of the search results, 
the users could get confused and frustrated to the point that they won't use the 
application.
\subsubsection{Cost}
\subsubsection*{Option 1: Single item view with swiping}
This option would be implemented using Android Swipe Views that allow the users 
to motion left and right to browse through the search results. The speed impact 
using this implementation would be the cost of rendering all the details for each 
item. 
\subsubsection*{Option 2: Item list view with pages}
Implementing this option would be done using the built-in Android ListView 
activity. Speed would be reliant on the android device and internet connectivity 
to render the list of items to be displayed. The speed impact would not be as 
significant as option 1 as only the basic details re rendered for each item.
\subsubsection*{Option 3: Item list view without pages}
The cost of implementation for this option would be similar to that of option 2. 
The only difference is that more search results would render automatically when 
users scroll down to the end of the already displayed results. 

\subsubsection{Evaluation}
All three options bring their own advantages to the table. The first option, will 
provide much more details regarding the items which is great. The drawback to 
option 1 is that it could potentially take much longer for the user to browse the 
hundreds of search results that are returned. Option 2 allows the user to set the 
number of results displayed per page. This allows the user to quickly browse through 
hundreds of items. Option 3 has some of the advantages of option 2, however users 
could get confused when they reach the end of the search results. The users would 
have to swipe down far enough for more options to be presented. If they don't, this 
could lead them to think that there are no more search results and think that the 
specific item that they are searching for is not available. 
\subsubsection{Best Choice}
The best choice for the app is to use Option 2 where the results are presented in 
a list view with X number of results presented per page. This result is very 
similar to other apps that the user would likely have used. It's such a 
straightforward approach, that even if the user hasn't used a similar 
application, there should not be any confusing when browsing the results. 

\subsection{Technology: Handling No Search Results}
\subsubsection{Options}

\subsubsection*{Option 1: Display message}
When there are no search results available, we could simply display a message 
that states ``There were no results found based off your search criteria. Try 
modifying your search criteria and try again.''
\subsubsection*{Option 2: Display alternate search results}
When there are no search results based on the provided criteria, we could 
modify the criteria and present search results for the modified search. When 
presenting these alternate search results, we would display a message that 
states what the search criteria is.
\subsubsection*{Option 3: Display alternate results based off search history}
If no items are returned for a specific search, the users could be presented 
with a history of their past searches. This could allow the users to revisit 
past searches that they may want to possibly continue.
\subsubsection{Use in design}
While not the most critical piece of the system, the functionality will be part 
of the ``ease to use'' aspect of the app. Letting the user know that their 
search returned no results would be helpful in that the user would not have to 
guess why there is nothing displayed on the screen. They could wrongfully 
assume things such as the app has crashed or internet connection is not 
letting items load correctly.
\subsubsection{Cost}
\subsubsection*{Option 1: Display message}
Cost of implementation would be minimal. A simple message notifying them of no 
search results would also be very Fast as it would not take much to render the 
message. 
\subsubsection*{Option 2: Display alternate search results}
This option would have a high implementation cost. We would need to determine 
the best method in which to modify the search criteria in a manner that still 
represents relevant results. This would likely require some sort of machine 
learning, and would get better with time. The speed cost of this feature would 
be significant due to having to modify the search criteria, and then 
re-querying with the updated search.
\subsubsection*{Option 3: Display alternate results based off search history}
With this option, the implementation would be dependent on the search history 
functionality being implemented. If search history is implemented properly, 
the cost of implementation would be between medium and high. With this option, 
we run the risk of presenting history that is no longer relevant to the user. 
\subsubsection{Evaluation}
Option 1 is the simplest approach that will accomplish the required task. Both options 2 and 3 would be great options, if there were more time for this project, and if they weren't dependent on other features that will be implemented concurrently.
\subsubsection{Best Choice}
Option 1 would be the best choice given the simplicity of the feature and the 
timeline that we have for the project. A simple message will be clear and get 
the point across. Displaying alternate search results would take more time, 
and we could potentially display results that the user does not care for.

\subsection{Technology: Home Page}
\subsubsection{Options}
\subsubsection*{Option 1: Simple Search Bar}
This option is the simplest of them all. The home page will be a single search 
bar that allows the user to immediately start searching for their items. 
\subsubsection*{Option 2: Search Bar with trending items}
This option will have the search bar along with items that are trending on eBay. 
This trending items will need to be filtered in the backend to only display College related merchandise to keep with the theme of the application. 
\subsubsection*{Option 3: Search Bar with Browsing History or related items}
With this option, when the user uses the application for the first time, they 
will receive the home page with only a search bar. After subsequent uses with 
search history, the home page will be populated with items that were previously 
viewed or returned in a user's search.  
\subsubsection{Use in design}
This will be an important piece of the application. It's going to be the first 
screen that the user sees when launching the application. A great first 
impression is always important thus we will need to make sure that the option 
that we go with does just that.
\subsubsection{Cost}
\subsubsection*{Option 1: Simple Search Bar}
The implementation costs for this option would be minimal. Implementing the 
search bar and ``Search'' button would consist of Android's built-in TextView 
and button within a single Activity. 
\subsubsection*{Option 2: Search Bar with trending items}
This feature would have a higher cost of implementation. First and foremost, we 
would need to determine the definition of ``Trending'' items. This alone could 
have a high cost of implementation. This option would also have medium speed 
costs as it would consist of querying for the trending items.
\subsubsection*{Option 3: Search Bar with Browsing History or related items}
This option would have the highest of the implementation costs. First we would 
need to have previous search history to allow to customize this page based off 
search history. Search history availability would be based off the ``Saved 
Search'' feature that will be implemented for this project. Speed cost would 
be medium as we would need to pull the search history for items viewed and then 
having to requery eBay for the details on these items. 
\subsubsection{Evaluation}
Option 1, is a very simple approach. Having only a search bar without any other 
distractions will allow the user to begin searching for their items immediately. 
Option 2 will be a neat feature to have, although it could present issues with 
our timeline. Option 3, will depend on the saved search features, it could prove 
to be too big of a dependency to implement this in a timely fashion.
\subsubsection{Best Choice}
Due to time constraints option 1 is the best solution. The home page will 
consist of a simple search bar allowing the user to start searching for items 
immediately. The background of the home page will consist of the UGear logo. 
The empty space would then be populated with a list view of items returned from the search.
\section{Conclusion}


\end{document}